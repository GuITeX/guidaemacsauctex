\documentclass[10pt,a4paper]{article}
% \title{L'arte di scrivere in \LaTeX{}\\
%   con\\
%   GNU Emacs \& AUC\TeX{}}
\title{Introduzione a GNU Emacs \& AUCTeX}
% \title{Una introduzione a GNU Emacs \& AUC\TeX{}}
% \title{}

% \author{Orlando Iovino  \thanks{Curatore di \emacs{} \& \auctex{} per Windows} 
% \and Pinco Pallino   \thanks{Curatore di \emacs{} \& \auctex{} per Linux}
% \and Tizio Sempronio \thanks{Curatore di \emacs{} \& \auctex{} per Mac OS}}

\author{Orlando Iovino\\(Windows)
  \and Pinco Pallino\\(GNU/Linux)
  \and Tizio Sempronio\\(Mac OS)}

\usepackage{macro}

\usepackage{lipsum} %%%%%%%%%%%%%%%% TEMP

\begin{document}
\maketitle
\begin{abstract}\sffamily
 % \lipsum[1] 
\end{abstract}

\begin{multicols}{2}
  \tableofcontents
\end{multicols}

\section{Introduzione}
\label{sec:intro}

%\lipsum[1]

\section{Installare \emacs}
\label{sec:install}

In questo paragrafo verranno date le informazioni necessarie per installare
\emacs{} sui tre principali sistemi operativi, vale a dire: 
\textsf{MicroSoft Windows (Xp, 7)},
\textsf{GNU/Linux} e %%% Specificare?
\textsf{Mac OS}.  %%% Specificare?


\subsection*{\emacs{} per Windows}
\label{sec:installwin}
Per dirla tutta, in un sistema operativo MicroSoft Windows (Xp, Seven)
non esiste una vera e propria installazione, pertanto per poter
\emph{installare} ed usare \emacs{} in questo ambiente,
è sufficiente procurarsi l'archivio contenente gli eseguibili del programma.

Per questa guida si è fatto riferimento al mirror ufficiale
(\href{http://ftp.gnu.org/pub/gnu/emacs/windows/}{\mano{} \emacs}),
da cui bisogna scaricare l'archivio compresso
\href{http://ftp.gnu.org/pub/gnu/emacs/windows/emacs-23.1-barebin-i386.zip}
{\textsf{emacs-23.1-barebin-i386.zip}}.

Una volta de-compresso, i file ottenuti devono essere copiati in una
posizione di comodo. Una buona soluzione, per esempio, è data da:

\begin{verbatim}
C:/Programmi/Emacs/
\end{verbatim}

La cartella appena creata sarà così formata:
\begin{center}\ttfamily
 \begin{tabular}{lrl}
  <DIR>&        & bin       \\
       &  1.044 & BUGS      \\
       & 35.147 & COPYING   \\
  <DIR>&        & etc       \\
  <DIR>&        & info      \\
       & 36.533 & INSTALL   \\
  <DIR>&        & leim      \\
  <DIR>&        & lisp      \\
       &  5.276 & README    \\
       & 11.350 & README.W32\\
  <DIR>&        & site-lisp \\
 \end{tabular}
\end{center}

Per creare un collegamento nel menù \textsf{Start}, bisogna
eseguire (doppio click) il file \emph{addpm.exe}. Attenzione però: con questa
procedura non viene aggiunto il percorso degli eseguibili alla  
variabile d'ambiente \textsf{PATH}. Quindi per poter lanciare 
\emacs{} da qualunque posizione, qualora si faccia un forte uso del prompt dei
comandi di Windows, questa operazione va fatta a mano.

\subsection*{\emacs{} per GNU/Linux}
\label{sec:installlinux}
\emacs{} fa parte del progetto GNU quindi è presente in tutti i sistemi
operativi della famiglia GNU/Linux. Se non è già installato, il metodo più
semplice per ottenere \emacs{} in un sistema GNU/Linux è far riferimento al
gestore pacchetti della propria distribuzione. Si possono naturalmente
utilizzare i gestori di pacchetti a interfaccia grafica, riportiamo qui inoltre
i comandi da terminale che possono essere eseguiti per installare \emacs{} in
alcune delle principali distribuzioni: Debian e Ubuntu:
\begin{verbatim}
$ sudo apt-get install emacs
\end{verbatim}
Fedora:
\begin{verbatim}
$ sudo yum install emacs
\end{verbatim}
OpenSUSE: %chiedere a qualcuno
\begin{verbatim}
$ sudo ..... 
\end{verbatim}

Il metodo più difficile, per i non avvezzi all'uso del terminale, consiste nel
compilare \emacs{} a partire dal codice sorgente, tuttavia non è intenzione di
questa guida spiegare come fare ciò.

Dopo averlo installato, \emacs{} potrà essere avviato facendo clic sul suo
lanciatore oppure eseguendo da terminale il comando
\begin{verbatim}
$ emacs
\end{verbatim}
Se si desidera utilizzare \emacs{} con interfaccia testuale bisogna aggiungere
l'opzione \verb|-nw| oppure \verb|--no-window-system|:
\begin{verbatim}
$ emacs -nw
\end{verbatim}
Possono essere aggiunti come argomenti da linea di comando il percorso del file
(o dei file) che si vogliono modificare:
\begin{verbatim}
$ emacs file1.tex file2.tex
\end{verbatim}

\subsection*{\emacs{} per Mac OS}
\label{sec:installmac}

\lipsum[1]

\section{Installare \auctex}
\label{sec:installauc}
%%%Dire cosa è AUCTEX?
\subsection*{\auctex{} per Windows}
\label{sec:auctexwin}

Le operazioni per rendere funzionante \auctex{} per \emacs{} in ambiente
Windows sono ancora più semplici.

Dal sito di riferimento di \auctex{} per Windows 
(\href{%
http://www.gnu.org/software/auctex/download-for-windows.html}%
{\mano{} \auctex}), bisogna scaricare l'archivio \href{%
http://ftp.gnu.org/pub/gnu/auctex/auctex-11.86-e23.2-msw.zip}%
{\textsf{auctex-11.86-e23.2-msw.zip}}, che altro non è che un insieme di
pacchetti precompilati.

Una volta de-compresso, è sufficiente copiare i file di tale archivio
nel punto dove abbiamo messo \emacs{}, cioè
\texttt{C:/Programmi/Emacs/}.

 Si noterà che esiste già una cartella
\textsf{site-lisp}: i file contenuti in  quella di \auctex{} vanno copiati
nella omonima di \emacs. In quest'ultima, quindi, dovrebbero
esserci cinque file in totale:

\begin{center}\ttfamily
 \begin{tabular}{lrl}
  <DIR>&       &        auctex\\
  <DIR>&       &  site-start.d\\
       &    189& site-start.el\\
       &    136&    subdirs.el\\
       & 14.395&   tex-site.el\\
 \end{tabular}
\end{center}

Aprendo successivamente \emacs{} si noterà qualche cambiamento
non appena si passa alla modalità \textsf{tex-mode} o a quella \textsf{latex-mode}.

\subsection*{\auctex{} per GNU/Linux}
\label{sec:auctexlinux}

Anche il pacchetto \auctex{} è presente nei sistemi GNU/Linux. Come già detto
per \emacs, per installare \auctex{} si consiglia di utilizzare il gestore
pacchetti della propria distribuzione. Ecco i comandi da usare nelle principali
distribuzioni: Debian:
\begin{verbatim}
$ sudo apt-get install auctex
\end{verbatim}
Fedora:
\begin{verbatim}
$ sudo yum install emacs-auctex
\end{verbatim}
OpenSUSE: % chiedere a qualcuno il nome del comando
\begin{verbatim}
$ sudo .... emacs-auctex
\end{verbatim}


\subsection*{\auctex{} per Mac OS}
\label{sec:auctexmac}

\lipsum[1]

\section{Personalizzare \emacs{} e \auctex}
\label{sec:personal}

\lipsum[1]

\section{Ref\TeX{}}
\label{sec:reftex}

\lipsum[1]

\section{Correttore ortografico}
\label{sec:corr}
Per abilitare \emacs{} a lavorare con la lingua italiana, bisogna
installare il correttore ortografico. Le procedure, poiché differenti,
vengono di seguito esposte per tutti e tre i sistemi operativi. 
L'uso è ovviamente invariante. 

Per poter fare una \emph{revisione} del proprio file, bisogna digitare:
\begin{verbatim}
M-x ispell
\end{verbatim}
in questo modo il dizionario proporrà di sequenza, la correzione
dei termini che trova errati; un secondo modo, per scrivere
correttamente, è attivare la modalità \emph{flyspell} (correzione al
volo, interattiva). Se la si vuole
attivare solo sul documento corrente basta andare in:
\begin{verbatim}
Tools > Spell Checking > Automatic spell checking (Flyspell)
\end{verbatim}
se invece la si vuole tenere sempre abilitata,
basta aggiungere la riga:
\begin{verbatim}
(add-hook 'LaTeX-mode-hook 'flyspell-mode)
\end{verbatim}
al proprio file \emph{.emacs}. Quando viene rilevata una parola
potenzialmente sbagliata, questa verrà visualizzata in rosso. La
correzione può essere fatta a mano o con la combinazione \verb!M-$!,
la quale proporrà possibili soluzioni.

\subsection*{Windows}
\label{sec:aspellwin}
Per installare il correttore ortografico in ambiente
Windows, si può usare il programma \textsf{GNU Aspell}
(\href{http://aspell.net/win32/}{\mano{} \textsf{GNU Aspell - Win32 version}}). L'ultima
versione stabile per Windows, seppur datata Dicembre 2002, è la \textsf{0.50-3}.

I file che bisogna procurarsi sono quelli che vengono di seguito
riportati; si badi bene ad installarli nell'ordine in cui vengono dati:

\begin{enumerate}
\item l'installer completo \href{http://ftp.gnu.org/gnu/aspell/w32/Aspell-0-50-3-3-Setup.exe}%
{\textsf{Aspell-0-50-3-3-Setup.exe}} (che rappresenta il programma
vero e proprio);
\item il dizionario precompilato per l'italiano
\href{http://ftp.gnu.org/gnu/aspell/w32/Aspell-it-0.50-2-3.exe}%
{\textsf{Aspell-it-0.50-2-3.exe}}.
\end{enumerate}

Alla stessa pagina di questi file sono disponibili tanti altri
dizionari precompilati, più o meno aggiornati, di altre lingue.

Resta qualche altra operazione da fare. Come prima cosa si deve copiare il
percorso della cartella dove sono contenuti gli eseguibili di Aspell nella
\textsf{PATH} di Windows. Secondo, aggiungere al proprio \emph{.emacs} le seguenti
righe di codice:

\begin{verbatim}
(setq-default ispell-program-name "aspell")
(setq-default ispell-extra-args '("--reverse"))
(setq ispell-dictionary "italiano")
\end{verbatim}
dove la seconda riga di codice serve a \emph{fixare} dei problemi con le versioni
precedenti.

% (setq flyspell-use-meta-tab nil)
%% DIRE A COSA SERVONO LE RIGHE

\subsection*{GNU/Linux}
\label{sec:aspelllinux}

Il correttore ortografico GNU Aspell può essere installato facilmente su
GNU/Linux utilizzando, come al solito, il gestore pacchetti della propria
distribuzione. Il dizionario italiano di GNU Aspell si chiama \verb|aspell-it|
quindi da terminale può essere installato in Debian con il comando
\begin{verbatim}
$ sudo apt-get install aspell-it
\end{verbatim}
in Fedora con
\begin{verbatim}
$ sudo yum install aspell-it
\end{verbatim}
mentre in OpenSUSE si può eseguire il comando: % chiedere qual è il comando
\begin{verbatim}
$ sudo ..... aspell-it
\end{verbatim}

\subsection*{Mac OS}
\label{sec:aspellmac}

\lipsum[1]

\section{Autocomplete mode}
\label{sec:auto}

\lipsum[1]

% Per GNU/Linux vedere
% http://elubuntu.blogspot.com/2010/04/autocompletamento-in-emacs-2.html

\section{Reference Card}
\label{sec:refcard}

Oltre alla vasta documentazione che si può reperire in rete, di 
particolare utilità sono le \emph{Reference Card}, disponibili sia per
\emacs{}, sia per \auctex{}.

Si tratta, come è facile immaginare, di una sorta di riassunto in
forma tabulare dei principali comandi per poter usare questi
programmi,  eventualmente  da stampare, e tenere a portata di mano
sulla  propria scrivania.

Le reference card sono disponibili sia in rete, sia sul proprio
computer una volta installati \emacs{} ed \auctex. Sul proprio
computer dovrebbero esserci anche i sorgenti pronti per essere compilati.

\section{Conclusioni}
\label{sec:fine}

\lipsum[1]


%% METTERE LA BIBLIOGRAFIA

\end{document}

%%% Local Variables: 
%%% mode: latex
%%% coding: utf-8
%%% TeX-master: t
%%% End: 
