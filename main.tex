\documentclass[10pt,a4paper]{article}

\title{Una introduzione a GNU Emacs \& AUC\TeX{}}
\author{Orlando Iovino\\(Windows)
  \and Pinco Pallino\\(GNU/Linux)
  \and Tizio Sempronio\\(Mac OS)}
\date{\today}

\usepackage{macro}
 
\usepackage{lipsum}

\begin{document} 
\maketitle
\begin{abstract}\sffamily
\noindent Questo articolo nasce con l'intenzione di dare una breve panoramica
all'uso di \emacs{} per \LaTeX{}, con l'ausilio di \auctex, per i
principali sistemi operativi. Dopo una introduzione ai programmi,
verranno esposte le modalità per poter installare e configurare
\emacs{} e \auctex{} per poter scrivere     
documenti \LaTeX{}. L'articolo non ha la pretesa di essere un
\emph{manuale}, bensì un piccolo tutorial per far avvicinare gli 
utenti di \LaTeX{} a questo programma, non così troppo complicato 
come spesso viene indicato.
\end{abstract}

\begin{multicols}{2}
  \tableofcontents
\end{multicols}

\section{Introduzione a GNU Emacs \& AUC\TeX}
\label{sec:intro}

\textcolor{red!50}{\lipsum[1]}

\section{Installare \emacs}
\label{sec:install}

In questo paragrafo verranno date le informazioni necessarie per installare
\emacs{} sui tre principali sistemi operativi, vale a dire: 
\textsf{MicroSoft Windows (Xp, 7)},
\textsf{GNU/Linux} e %%% Specificare?
\textsf{Mac OS}.  %%% Specificare?

\subsection*{\emacs{} per Windows}
\label{sec:installwin}
Per dirla tutta, in un sistema operativo MicroSoft Windows (Xp, Seven)
non esiste una vera e propria installazione, pertanto per poter
\emph{installare} ed usare \emacs{} in questo ambiente,
è sufficiente procurarsi l'archivio contenente gli eseguibili del programma.

Per questa guida si è fatto riferimento al mirror ufficiale
(\href{http://ftp.gnu.org/pub/gnu/emacs/windows/}{\mano{} \emacs}),
da cui bisogna scaricare l'archivio compresso
\href{http://ftp.gnu.org/pub/gnu/emacs/windows/emacs-23.1-barebin-i386.zip}
{\textsf{emacs-23.1-barebin-i386.zip}}.

Una volta de-compresso, i file ottenuti devono essere copiati in una
posizione di comodo. Una buona soluzione, per esempio, è data da:

\begin{Verbatim}
C:/Programmi/Emacs/
\end{Verbatim}

La cartella appena creata sarà così formata:
\begin{center}\ttfamily
 \begin{tabular}{lrl}
  <DIR>&        & bin       \\
       &  1.044 & BUGS      \\
       & 35.147 & COPYING   \\
  <DIR>&        & etc       \\
  <DIR>&        & info      \\
       & 36.533 & INSTALL   \\
  <DIR>&        & leim      \\
  <DIR>&        & lisp      \\
       &  5.276 & README    \\
       & 11.350 & README.W32\\
  <DIR>&        & site-lisp \\
 \end{tabular}
\end{center}

Per creare un collegamento nel menù \textsf{Start}, bisogna
eseguire (doppio click) il file \emph{addpm.exe}. Attenzione però: con questa
procedura non viene aggiunto il percorso degli eseguibili alla  
variabile d'ambiente \textsf{PATH}. Quindi per poter lanciare 
\emacs{} da qualunque posizione, qualora si faccia un forte uso del prompt dei
comandi di Windows, questa operazione va fatta a mano.

\subsection*{\emacs{} per GNU/Linux}
\label{sec:installlinux}
\emacs{} fa parte del progetto GNU quindi è presente in tutti i sistemi
operativi della famiglia GNU/Linux. Se non è già installato, il metodo più
semplice per ottenere \emacs{} in un sistema GNU/Linux è far riferimento al
gestore pacchetti della propria distribuzione. Si possono naturalmente
utilizzare i gestori di pacchetti a interfaccia grafica, riportiamo qui inoltre
i comandi da terminale che possono essere eseguiti per installare \emacs{} in
alcune delle principali distribuzioni: Debian e Ubuntu:
\begin{Verbatim}
$ sudo apt-get install emacs
\end{Verbatim}
Fedora:
\begin{Verbatim}
$ sudo yum install emacs
\end{Verbatim}
OpenSUSE:
\begin{Verbatim}
$ sudo zypper install emacs
\end{Verbatim} 

Il metodo più difficile, per i non avvezzi all'uso del terminale, consiste nel
compilare \emacs{} a partire dal codice sorgente, tuttavia non è intenzione di
questa guida spiegare come fare ciò.

Dopo averlo installato, \emacs{} potrà essere avviato facendo clic sul suo
lanciatore oppure eseguendo da terminale il comando
\begin{Verbatim}
$ emacs
\end{Verbatim}
Se si desidera utilizzare \emacs{} con interfaccia testuale bisogna aggiungere
l'opzione \verb|-nw| oppure \verb|--no-window-system|:
\begin{Verbatim}
$ emacs -nw
\end{Verbatim}
Possono essere aggiunti come argomenti da linea di comando il percorso del file
(o dei file) che si vogliono modificare:
\begin{Verbatim}
$ emacs file1.tex file2.tex
\end{Verbatim}

\subsection*{\emacs{} per Mac OS}
\label{sec:installmac}

\textcolor{red!50}{\lipsum[1]}

\section{Installare \auctex}
\label{sec:installauc}

\subsection*{\auctex{} per Windows}
\label{sec:auctexwin}

Le operazioni per rendere funzionante \auctex{} per \emacs{} in ambiente
Windows sono ancora più semplici.

Dal sito di riferimento di \auctex{} per Windows 
(\href{%
http://www.gnu.org/software/auctex/download-for-windows.html}%
{\mano{} \auctex}), bisogna scaricare l'archivio \href{%
http://ftp.gnu.org/pub/gnu/auctex/auctex-11.86-e23.2-msw.zip}%
{\textsf{auctex-11.86-e23.2-msw.zip}}, che altro non è che un insieme di
pacchetti precompilati.

Una volta de-compresso, è sufficiente copiare i file di tale archivio
nel punto dove abbiamo messo \emacs{}, cioè
\texttt{C:/Programmi/Emacs/}.

 Si noterà che esiste già una cartella
\textsf{site-lisp}: i file contenuti in  quella di \auctex{} vanno copiati
nella omonima di \emacs. In quest'ultima, quindi, dovrebbero
esserci cinque file in totale:

\begin{center}\ttfamily
 \begin{tabular}{lrl}
  <DIR>&       &        auctex\\
  <DIR>&       &  site-start.d\\
       &    189& site-start.el\\
       &    136&    subdirs.el\\
       & 14.395&   tex-site.el\\
 \end{tabular}
\end{center}

Aprendo successivamente \emacs{} si noterà qualche cambiamento
non appena si passa alla modalità \textsf{tex-mode} o a quella \textsf{latex-mode}.

\subsection*{\auctex{} per GNU/Linux}
\label{sec:auctexlinux}

Anche il pacchetto \auctex{} è presente nei sistemi GNU/Linux. Come già detto
per \emacs, per installare \auctex{} si consiglia di utilizzare il gestore
pacchetti della propria distribuzione. Ecco i comandi da usare per
installare da terminale il pacchetto nelle principali distribuzioni: Debian e Ubuntu:
\begin{Verbatim}
$ sudo apt-get install auctex
\end{Verbatim}
Fedora:
\begin{Verbatim}
$ sudo yum install emacs-auctex
\end{Verbatim}
OpenSUSE:
\begin{Verbatim}
$ sudo zypper install emacs-auctex
\end{Verbatim}


\subsection*{\auctex{} per Mac OS}
\label{sec:auctexmac}

\textcolor{red!50}{\lipsum[1]}

\section{Personalizzare \emacs{} e \auctex}
\label{sec:personal}

\textcolor{red!50}{\lipsum[1]}

\section{Ref\TeX{}}
\label{sec:reftex}

\textcolor{red!50}{\lipsum[1]}

\section{Correttore ortografico}
\label{sec:corr}
Per abilitare \emacs{} a lavorare con la lingua italiana, bisogna
installare il correttore ortografico. Le procedure, poiché differenti,
vengono di seguito esposte per tutti e tre i sistemi operativi. 
L'uso è ovviamente invariante. 

Per poter fare una \emph{revisione} del proprio file, bisogna digitare:
\begin{Verbatim}
M-x ispell
\end{Verbatim}
in questo modo il dizionario proporrà in sequenza, la correzione
dei termini che trova errati; un secondo modo, per scrivere
correttamente, è attivare la modalità \emph{flyspell} (correzione al
volo, interattiva). Se la si vuole
attivare solo sul documento corrente basta andare in:
\begin{Verbatim}
Tools > Spell Checking > Automatic spell checking (Flyspell)
\end{Verbatim}
se invece la si vuole tenere sempre abilitata,
basta aggiungere la riga:
\begin{Verbatim}
(add-hook 'LaTeX-mode-hook 'flyspell-mode)
\end{Verbatim}
al proprio file \emph{.emacs}. Quando viene rilevata una parola
potenzialmente sbagliata, questa verrà visualizzata in rosso. La
correzione può essere fatta a mano o con la combinazione \verb!M-$!,
la quale proporrà possibili soluzioni.

\subsection*{Windows}
\label{sec:aspellwin}
Per installare il correttore ortografico in ambiente
Windows, si può usare il programma \textsf{GNU Aspell}
(\href{http://aspell.net/win32/}{\mano{} \textsf{GNU Aspell - Win32 version}}). L'ultima
versione stabile per Windows, seppur datata Dicembre 2002, è la \textsf{0.50-3}.

I file che bisogna procurarsi sono quelli che vengono di seguito
riportati; si badi bene ad installarli nell'ordine in cui vengono dati:

\begin{enumerate}
\item l'installer completo \href{http://ftp.gnu.org/gnu/aspell/w32/Aspell-0-50-3-3-Setup.exe}%
{\textsf{Aspell-0-50-3-3-Setup.exe}} (che rappresenta il programma
vero e proprio);
\item il dizionario precompilato per l'italiano
\href{http://ftp.gnu.org/gnu/aspell/w32/Aspell-it-0.50-2-3.exe}%
{\textsf{Aspell-it-0.50-2-3.exe}}.
\end{enumerate}

Alla stessa pagina di questi file sono disponibili tanti altri
dizionari precompilati, più o meno aggiornati, di altre lingue.

Resta qualche altra operazione da fare. Come prima cosa si deve copiare il
percorso della cartella dove sono contenuti gli eseguibili di Aspell nella
\textsf{PATH} di Windows. Secondo, aggiungere al proprio \emph{.emacs} le seguenti
righe di codice:

\begin{Verbatim}
(setq-default ispell-program-name "aspell")
(setq-default ispell-extra-args '("--reverse"))
(setq ispell-dictionary "italiano")
\end{Verbatim}
dove la seconda riga di codice serve a \emph{fixare} dei problemi con le versioni
precedenti.

\subsection*{GNU/Linux}
\label{sec:aspelllinux}

Il correttore ortografico GNU Aspell può essere installato facilmente su
GNU/Linux utilizzando, come al solito, il gestore pacchetti della propria
distribuzione. Il dizionario italiano di GNU Aspell si chiama \verb|aspell-it|
quindi da terminale può essere installato in Debian con il comando
\begin{Verbatim}
$ sudo apt-get install aspell-it
\end{Verbatim}
in Fedora con
\begin{Verbatim}
$ sudo yum install aspell-it
\end{Verbatim}
mentre in openSUSE si può eseguire il comando:
\begin{Verbatim}
$ sudo zypper install aspell-it
\end{Verbatim}

\subsection*{Mac OS}
\label{sec:aspellmac}

\textcolor{red!50}{\lipsum[1]}

\section{Ricerca diretta-inversa}
\label{sec:ricdirinv}

Una delle  principali funzioni che spesso l'utente  \LaTeX{} cerca nel
proprio editor, è quella della ricerca inversa-diretta, molto utile in
fase di revisione finale del proprio documento, e specie quando il
documento è abbastanza corposo.

Alcuni editor di  testo hanno un visualizzatore integrato preconfigurato  
per la  ricerca dal  sorgente al  pdf e  viceversa, come  per
esempio \TeX works; altri,  come \emacs, richiedono alcuni passaggi per
arrivare allo scopo.

Le procedure  per abilitare tale funzione  in \emacs{} variano  per i tre
sistemi  operativi, ma non  solo: variano  anche i  visualizzatori che
permettono tale funzione. In~\parencite{arte:latex} vengono citate le
cosiddette \emph{coppie vincenti} adatte allo scopo.

In ambiente  Windows, l'unico visualizzatore  capace di fare  questo è
SumatraPDF, in  ambiente Linux si può  usare [\ldots],
mentre in ambiente Mac OS si usa [\ldots].

\subsection*{Ricerca diretta-inversa su Windows}
\label{sec:fiswin}
\textcolor{red}{%
Prima di entrare nel vivo dell'argomento, bisogna ovviare ad un bug che
da tempo segue il programma \emacs{} su Windows [\ldots]}

Per i sistemi operativi Windows il solo visualizzatore capace di
interagire con il file sorgente \textsf{.tex} creato con \emacs{} è
\textsf{SumatraPDF} (\href{http://blog.kowalczyk.info/software/sumatrapdf/free-pdf-reader.html}{\mano{}~SumatraPDF}).

Per la verità, il visualizzatore sviluppato da Krzysztof Kowalczyk non
supporta questa funzionalità, o meglio, non è ancora accessibile
all'utente, pertanto, dopo averlo installato si può
procedere nel seguente modo:
\begin{enumerate}
\item si scarica l'eseguibile \textsf{SumatraPDF-TeX}
  (\href{http://william.famille-blum.org/software/sumatra/}{\mano{}~SumatraPDF
    for \TeX\ users}) di William Blum
  che ha sviluppato appunto la funzionalità di ricerca diretta-inversa
  e fornito un modo per accedervi;
\item si rinomina l'eseguibile SumatraPDF-TeX in SumatraPDF;
\item si copia tale file nella cartella dove è contenuto quello
  installato precedentemente.
\end{enumerate}

A queste operazioni segue la configurazione di \textsf{SumatraPDF} e di \emacs,
mediante il proprio file \emph{.emacs}.

Per la ricerca inversa si deve aprire SumatraPDF, andare in
\texttt{Impostazioni > Opzioni...} e digitare la riga
\begin{verbatim}
C:/Programmi/emacs/bin/emacsclientw.exe +%l "%f"
\end{verbatim}
in \emph{Set inverse search command-line}.

Per la ricerca diretta invece, si deve scaricare lo script
\textsf{sumatra-forward} (\href{http://william.famille-blum.org/software/sumatra/sumatra-forward.el}{\mano{}~sumatra-forward.el})
e copiarlo in una cartella accessibile a \emacs{} (l'autore delle note
l'ha posizionato in \texttt{C:/Programmi/emacs/bin/site-lisp}), per
poi richiamarlo all'interno del proprio file \emph{.init} con la riga
\begin{verbatim}
(require 'sumatra-forward)
\end{verbatim}
ed infine accertarsi di avere nella cartella degli eseguibili di
\emacs, l'applicazione \textsf{ddeclient}, altrimenti scaricabile da
[].

Fatto questo resta da far capire a \emacs{} che si vuole la ricera
diretta inversa, e per farlo si possono scrivere nel proprio \emph{.init}
file, le seguenti righe:
\begin{verbatim}
(custom-set-variables
 '(TeX-source-correlate-method (quote synctex))
 '(TeX-source-correlate-mode t)
 '(TeX-source-correlate-start-server t)
)
\end{verbatim}

Per la ricerca inversa basta premere il tasto \textsf{F8} e nel
visualizzatore verrà evidenziata la riga corrispondente; per le
ricerca diretta invece, basta fare doppio click nel visualizzatore e
nel file .tex il cursore si sposterà all'inizio della riga che
contiene quella parola.

Si vuole infine dare un suggerimento al lettore, qualora \textsf{SumatraPDF}
non sia il proprio visualizzatore preferito. Infatti è possibile
impostare \textsf{SumatraPDF} come visualizzatore predefinito di \emacs{} e non
del sistema operativo (lasciando magari \textsf{Adobe Reader}), aggiungendo le
seguenti righe a quelle sopra descritte:
\begin{verbatim}
 '(TeX-view-program-list (quote (("SumatraPDF" "SumatraPDF %o"))))
 '(TeX-view-program-selection (quote (
   (output-pdf "SumatraPDF") 
   (output-dvi "Yap") 
   (output-html "start"))))
\end{verbatim}

\subsection*{Ricerca diretta-inversa su GNU/Linux}
\label{sec:fislinux}

\textcolor{red}{\lipsum[1]}

\subsection*{Ricerca diretta-inversa su Mac OS}
\label{sec:fismac}

\textcolor{red}{\lipsum[1]}

\section{Autocomplete mode}
\label{sec:auto}

\textcolor{red!50}{\lipsum[1]}

% Per GNU/Linux vedere
% http://elubuntu.blogspot.com/2010/04/autocompletamento-in-emacs-2.html

\section{Reference Card}
\label{sec:refcard}

Oltre alla vasta documentazione che si può reperire in rete ed ai
manuali ufficiali, di particolare utilità sono le \emph{Reference
  Card}, disponibili sia per \emacs{}, sia per \auctex{}.

Si tratta di un elenco delle principali funzionalità e dei comandi 
a queste associate per poter, all'occorrenza, ricordare e trovare
velocemente quello che serve.

Il tutto è riportato in maniera ordinata e concisa in una sorta di
forma tabulare, eventualmente  da stampare, e tenere a portata di mano
sulla propria scrivania.

Le reference card sono disponibili sia in rete, sia sul proprio
computer una volta installati \emacs{} ed \auctex. Sul proprio
computer dovrebbero esserci anche i sorgenti pronti per essere compilati.

\section{Conclusioni}
\label{sec:fine}

\textcolor{red!50}{\lipsum[1]}

\cleardoublepage
\addcontentsline{toc}{section}{\bibname}
\printbibliography

\end{document}

%%% Local Variables: 
%%% mode: latex
%%% coding: utf-8
%%% TeX-master: t
%%% End: 
