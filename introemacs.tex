%%  Copyright (C) 2012 M. Giordano, O. Iovino, M. Leccardi
%%
%%  This program is free software; you can redistribute it and/or modify
%%  it under the terms of the GNU General Public License as published by
%%  the Free Software Foundation; either version 3 of the License, or
%%  (at your option) any later version.
%%
%%  This program is distributed in the hope that it will be useful,
%%  but WITHOUT ANY WARRANTY; without even the implied warranty of
%%  MERCHANTABILITY or FITNESS FOR A PARTICULAR PURPOSE.  See the
%%  GNU General Public License for more details.

\documentclass[b5paper,11pt,oneside]{guidatematica}
\ProvidesFile{introemacs.tex}[2012/12/20 v.0.0 Breve guida all'uso di Emacs per LaTeX]

\usepackage{preambolo}
\usepackage{lipsum}

\begin{document}

\frontmatter*

\title{Una introduzione a GNU Emacs \& AUC\TeX{}}
\author{Mosè Giordano, Orlando Iovino, Matteo Leccardi}
\GetFileInfo{introemacs.tex}
\date{\filename\ \fileversion\ del \filedate}
\maketitle

\chapter{Licenza d'uso}
\label{chap:licenza}

Quest'opera è soggetta alla Creative Commons Public License versione 3.0 o
posteriore: \emph{Attribuzione}, \emph{Non Commerciale}, \emph{Condividi allo
  stesso modo}. L'enunciato integrale della licenza è reperibile sul sito
ufficiale \url{http://creativecommons.org/licenses/by-nc-sa/3.0/deed.it}.

\bigskip\noindent\textsc{tu sei libero:}
\begin{itemize}
\item Di riprodurre, distribuire, comunicare al pubblico, esporre in pubblico,
      rappresentare, eseguire e recitare quest'opera.
\item Di modificare quest'opera.
\end{itemize}
\textsc{alle seguenti condizioni:}
\begin{itemize}
\item[\Large\ccAttribution] Devi attribuire la paternità dell'opera nei modi indicati
      dall'autore o da chi ti ha dato l'opera in licenza e in modo tale da non
      suggerire che essi avallino te o il modo in cui tu usi l'opera.
\item[\Large\ccNonCommercial] Non puoi usare quest'opera per fini commerciali.
\item[\Large\ccShareAlike] Se alteri o trasformi quest'opera, o se la usi per crearne
      un'altra, puoi distribuire l'opera risultante solo con una licenza
      identica o equivalente a questa.
\end{itemize}

\chapter{Presentazione} 
\label{chap:presentazione}

Questa guida tematica presenta il programma GNU Emacs ed il pacchetto
\auctex{} per i principali sistemi operativi: Windows, GNU/Linux e Mac
OS. Dopo una introduzione al programma, si spiega come installare (e
configurare alcune funzioni) \emacs{} e \auctex{} per scrivere
documenti \LaTeX. Per finire si dà qualche accenno per personalizzare e
rendere più efficiente \emacs.

Si tenga presente che la guida è ben lontana dall'essere esauriente
per un editore di testi così importante come Emacs. Per ogni altro
approfondimento non riportato in queste righe rimandiamo il lettore ai
manuali ufficiali, in particolare a \cite{emacs:stallman} e
\cite{auctex:manual}. Speriamo, con queste brevi note, di convincere
il lettore che Emacs non è poi così complicato come spesso viene
indicato.

Infine, qualche parola sulla composizione di questa guida tematica,
scritta a \emph{sei mani} grazie sistema di controllo di versione Git
(\url{http://git-scm.com}). Chi vuole collaborare può ``clonare'' il
progetto dal repository ufficiale presente sul sito
\url{http://gitorious.org/} alla pagina
\url{https://gitorious.org/emacs-tutorial}. È inoltre gradita la
segnalazione di errori, refusi, commenti e suggerimenti o ogni altra
forma di collaborazione.

\begin{tabular}{cc}
\hspace{.34\textwidth} &                                                              \\
                       & \textsc{Mosè Giordano}                                       \\
                       & \textcolor{red}{\texttt{nome dot cognome at dominio dot it}} \\[1.5ex]
                       & \textsc{Orlando Iovino}                                      \\
                       & \texttt{orlando dot iovino at yahoo dot it}                  \\[1.5ex]
                       & \textsc{Matteo Leccardi}                                     \\ 
                       & \textcolor{red}{\texttt{nome dot cognome at dominio dot it}} \\
\end{tabular}


\newpage\tableofcontents*

\mainmatter

\chapter{Introduzione}
\label{chap:intro}

Lo sviluppo di \emacs{} è iniziato negli anni 1970 ai laboratori di Intelligenza
Artificiale del MIT, cioè molto prima che si diffondesse l'informatica che
conosciamo oggi.  Ciò comporta che \emacs{} utilizza una terminologia spesso
diversa da quella utilizzata dalla stragrande maggioranza degli altri programmi
che usiamo quotidianamente.  Per questo motivo non bisogna sorprendersi se
l'operazione di \emph{tagliare} del testo (generalmente indicata in inglese con
\emph{cutting}) in \emacs{} viene chiamata \emph{killing} e la successiva
operazione di \emph{incollare} il testo copiato (in inglese \emph{pasting})
viene chiamata \emph{yanking}.

\emacs{} è famoso per rendere possibile l'esecuzione di qualsiasi operazione
usando solo la tastiera, cioè senza l'ausilio del mouse.  A ogni comando può
essere associata una combinazione di tasti.  Sebbene questo fatto possa
inizialmente spaventare, è un vero punto di forza di \emacs{}, perché permette
di eseguire delle operazioni in maniera rapida, invece di dover spostare il
mouse alla ricerca di appositi bottoni.  In questa guida seguiremo la notazione
diffusa nel mondo di \emacs{} per indicare i tasti.  Quindi il tasto
\keys{\ctrl} verrà indicato con \verb!C!, l'\keys{Invio} viene indicato con
\verb!RET!, mentre \verb!M! indica il tasto \keys{\Meta}.  Sulle odierne
tastiere questo tasto è scomparso, si può ottenere lo stesso risultato con il
tasto \keys{\Alt} oppure premendo e rilasciando il tasto \keys{\esc}.  Nella
notazione di queste scorciatoie, un trattino fra due tasti all'interno di una
combinazione di tasti indica che i tasti vanno premuti contemporaneamente.
Così, quando si dirà che per eseguire un comando bisogna usare la combinazione
\verb!M-x! si dovranno premere contemporaneamente il tasto \keys{\Alt} e la
lettera \keys{x} oppure premere il tasto \keys{\esc}, rilasciarlo e premere la
lettera \keys{x}.

Il manuale di \emacs{}~\cite{emacs:stallman} può essere consultato all'interno
dello stesso editor di testo con le combinazioni di tasti \verb!C-h r! (premere
contemporaneamente il tasto \keys{\ctrl} e la lettera \keys{h}, rilasciare
questi tasti e premere la lettera \keys{r}) oppure \verb!C-h i d m Emacs RET!
(premere contemporaneamente il tasto \keys{\ctrl} e la lettera \keys{h},
rilasciare questi tasti e premere la lettera \keys{i}, poi la lettera \keys{d},
poi la lettera \keys{m}, poi scrivere \verb!Emacs! quindi premere il tasto
\keys{Invio}).  Se si preferisce, il manuale può anche essere consultato dal
menu \menu{Help > Read the Emacs Manual}.

% Mettere una tabella con le principali scorciatoie e corrispondenti comandi?

Una delle caratteristiche principali di \emacs{} è che è un editor di testo
personalizzabile in ogni suo aspetto ed espandibile, in modo da renderlo adatto
alle proprie esigenze.  Il linguaggio utilizzato per espandere \emacs{}
è l'Elisp, un dialetto del Lisp sviluppato appositamente per \emacs{}.  Esistono
migliaia di pacchetti scritti in Emacs Lisp che permettono di estendere
ulteriormente le funzionalità di questo editor.

La versione standard di \emacs{}, senza pacchetti aggiuntivi, fornisce già un
discreto supporto alla creazione di documenti \LaTeX, ma il pacchetto \auctex{}
mette a disposizione numerosi strumenti che rendono senza alcun dubbio \emacs{}
uno degli editor di testo più potenti per la scrittura di documenti \LaTeX.  In
questa guida vedremo alcune delle funzioni principali di \auctex.

%% --------------------------------------------------------------------------
%% Windows
%% --------------------------------------------------------------------------

\chapter{Emacs e AUC\TeX{}  su Windows}
\label{oi:chap:windows}

\epigraph{(\textit{a cura di} Orlando Iovino)}

\section{Installare Emacs}
\label{oi:sec:installemacs}

Per sistemi operativi Microsoft Windows non esiste una vera e propria procedura
di installazione. {\emacs} si può utilizzare semplicemente procurandosi
l'archivio che contiene gli eseguibili.

All'indirizzo \url{http://ftp.gnu.org/pub/gnu/emacs/windows/} si scarica
l'archivio compresso \zipname{emacs-24.2-bin-i386.zip} e lo si decomprime in una
posizione di comodo; una buona soluzione, per esempio, è decomprimere il
suddetto archivio in \directory{C:/Programmi/emacs/}.

Se si vuole creare un collegamento di \emacs{} nel menù Start, è
sufficiente un doppio clic sul file \emph{addpm.exe} contenuto nella
cartella \directory{C:/Programmi/emacs/bin}.

%%
%% 2012/08/19: Verificare questa affermazione, su Windows 7 è così.
%%
A questo punto \emacs{} è pronto per scrivere documenti in \LaTeX; l'utente
infatti non deve specificare nessun percorso circa gli eseguibili della
distribuzione installata: sarà \emacs{} stesso a trovarli.

\section{Installare AUC\TeX}
\label{oi:sec:installauctex}

All'indirizzo \url{http://www.gnu.org/software/auctex/download-for-windows}
si scarica l'archivio compresso \zipname{auctex-11.86-e23.2-msw.zip} e lo si
decomprime in \directory{C:/Programmi/emacs/}.  Sia nell'archivio di \auctex{}
che quello di \emacs, posizionato in precedenza nel percorso suddetto, vi è una
cartella dal nome \emph{site-lisp}: i file contenuti in queste cartelle vanno
uniti in una sola cartella.

\section{Correttore ortografico}
\label{oi:sec:aspell}

Per usare il correttore ortografico in \emacs, su un sistema operativo Windows,
si può installare il programma GNU Aspell disponibile all'indirizzo
\url{http://aspell.net/win32/}. L'ultima versione stabile per Windows, seppur
datata Dicembre 2002, è la 0.50-3.

I file che bisogna procurarsi sono quelli che vengono di seguito
riportati; si badi bene ad installarli nell'ordine in cui vengono
dati:

\begin{enumerate}
\item il programma completo
      \href{http://ftp.gnu.org/gnu/aspell/w32/Aspell-0-50-3-3-Setup.exe}%
      {Aspell-0-50-3-3-Setup.exe} (che rappresenta il programma vero e proprio);
\item il dizionario precompilato per l'italiano
      \href{http://ftp.gnu.org/gnu/aspell/w32/Aspell-it-0.50-2-3.exe}%
      {Aspell-it-0.50-2-3.exe}.
\end{enumerate}

Resta qualche altra operazione da fare. Come prima cosa si deve copiare il
percorso della cartella dove sono contenuti gli eseguibili di Aspell nella
variabile d'ambiente \textsf{PATH} di Windows. Secondo, aggiungere al proprio
\emph{.emacs} le seguenti righe di codice:
\begin{Verbatim}
(setq-default ispell-program-name "aspell")
(setq-default ispell-extra-args '("--reverse"))
(setq ispell-dictionary "italiano")
\end{Verbatim}
dove, la prima e la terza sono quelle strettamente necessarie, mentre la seconda
riga serve a risolvere dei problemi con le versioni precedenti di
\emacs.


\section{Ricerca diretta inversa}
\label{oi:sec:ricdirinv}

Prima di entrare nel vivo dell'argomento, bisogna ovviare ad un bug che da tempo
segue il programma \emacs{} su Windows, in particolare i sistemi Windows Vista e
Seven (su Xp non dovrebbero esserci problemi), ovvero il problema che si ha
quando si abilita il server in \emacs.

%%
%% 2012/09/05: SPECIFICARE QUESTA COSA
%%

La soluzione a tale problema è riportata in~\cite{emacsW32:wiki}
alla voce
\href{http://www.emacswiki.org/emacs/EmacsW32#toc49}{\emph{Waiting for
    Emacs server to start" is ownership problem on
    ~/.emacs.d/server}}.

Per i sistemi operativi Windows il solo visualizzatore capace di
interagire con il file sorgente \textsf{.tex} creato con \emacs{} è
SumatraPDF %
\sito{http://blog.kowalczyk.info/software/sumatrapdf/free-pdf-reader.html}%
{SumatraPDF}.

Per la verità, il visualizzatore sviluppato da Krzysztof Kowalczyk non
supporta questa funzionalità, o meglio, non è ancora accessibile
all'utente, pertanto, dopo averlo installato si può procedere nel
seguente modo:
\begin{enumerate}
\item si scarica l'eseguibile Sumatra-\TeX{} %
\sito{http://william.famille-blum.org/software/sumatra/}%
     {SumatraPDF for \TeX\ users}
di William Blum che ha sviluppato appunto la funzionalità di ricerca
diretta-inversa e fornito un modo per accedervi;
\item si rinomina l'eseguibile SumatraPDF-TeX in SumatraPDF;
\item si copia tale file nella cartella dove è contenuto quello
  installato precedentemente.
\end{enumerate}

A queste operazioni segue la configurazione di SumatraPDF e di \emacs,
mediante il proprio file \emph{.emacs}.

Per la ricerca inversa si deve aprire SumatraPDF, andare in
\texttt{Impostazioni} e quindi scegliere \texttt{Opzioni...} e
scrivere la riga
\begin{Verbatim}
C:/Programmi/emacs/bin/emacsclientw.exe +%l "%f"
\end{Verbatim}
in \emph{Set inverse search command-line}. La figura~\ref{oi:fig:sumatra:setup}
vale più di mille parole.

\begin{figure}[t]
  \centering
  \includegraphics[width=0.50\textwidth]{sumatrapdf}
  \caption{Impostazioni per il visualizzatore SumatraPDF.}
  \label{oi:fig:sumatra:setup}
\end{figure}

Per la ricerca diretta si deve scaricare lo script
sumatra-forward %
\sito{http://william.famille-blum.org/software/sumatra/sumatra-forward.el}
{sumatra-forward.el} %
e copiarlo in una cartella accessibile a \emacs{} per esempio
\directory{C:/Programmi/emacs/bin/site-lisp}  per poi richiamarlo all'interno
del proprio file \emph{.init} con la riga
\begin{Verbatim}
(require 'sumatra-forward)
\end{Verbatim}
ed infine accertarsi di avere nella cartella degli eseguibili di
\emacs, l'applicazione \textsf{ddeclient}, altrimenti scaricabile da %
\url{http://ftp.gnu.org/old-gnu/emacs/windows/docs/ntemacs/contrib/ddeclient.zip}.

Fatto questo resta da far capire a \emacs{} che si vuole la ricera
diretta inversa, e per farlo si possono scrivere nel proprio
\emph{.init} file, le seguenti righe:
\begin{Verbatim}
(setq TeX-source-correlate-method (quote synctex))
(setq TeX-source-correlate-mode t)
(setq TeX-source-correlate-start-server t)
\end{Verbatim}

Per la ricerca inversa basta premere il tasto \keys{F8} e nel visualizzatore
verrà evidenziata la riga corrispondente; per le ricerca diretta invece, basta
fare doppio click nel visualizzatore e nel file .tex il cursore si sposterà
all'inizio della riga che contiene quella parola.

Si vuole infine dare un suggerimento al lettore, qualora
\textsf{SumatraPDF} non sia il proprio visualizzatore
preferito. Infatti è possibile impostare \textsf{SumatraPDF} come
visualizzatore predefinito di \emacs{} e non del sistema operativo
(lasciando magari \textsf{Adobe Reader}), aggiungendo le seguenti
righe a quelle sopra descritte:
\begin{Verbatim}
(setq TeX-view-program-list (quote (("SumatraPDF" "SumatraPDF %o"))))
(setq TeX-view-program-selection (quote ( (output-pdf "SumatraPDF")
(output-dvi "Yap") (output-html "start"))))
\end{Verbatim}


%% --------------------------------------------------------------------------
%% GNU/Linux
%% --------------------------------------------------------------------------

\chapter{Emacs e AUC\TeX{} su GNU/Linux}
\label{mg:chap:linux}

\epigraph{(\textit{a cura di} Mosè Giordano)}

\section{Installare Emacs}
\label{mg:sec:installemacs}

\emacs{} fa parte del progetto GNU quindi è presente in tutti i
sistemi operativi della famiglia GNU/Linux. Se non è già installato,
il metodo più semplice per ottenere \emacs{} in un sistema GNU/Linux è
far riferimento al gestore pacchetti della propria distribuzione. Si
possono naturalmente utilizzare i gestori di pacchetti a interfaccia
grafica, riportiamo qui inoltre i comandi da terminale che possono
essere eseguiti per installare \emacs{} in alcune delle principali
distribuzioni: Debian e Ubuntu:
\begin{Verbatim}
$ sudo apt-get install emacs
\end{Verbatim}
% $
Fedora:
\begin{Verbatim}
$ sudo yum install emacs
\end{Verbatim}
% $
OpenSUSE:
\begin{Verbatim}
$ sudo zypper install emacs
\end{Verbatim}
% $

Il metodo più difficile, per i non avvezzi all'uso del terminale,
consiste nel compilare \emacs{} a partire dal codice sorgente,
tuttavia non è intenzione di questa guida spiegare come fare ciò.

Dopo averlo installato, \emacs{} potrà essere avviato facendo clic sul
suo lanciatore oppure eseguendo da terminale il comando
\begin{Verbatim}
$ emacs
\end{Verbatim}
% $
Se si desidera utilizzare \emacs{} con interfaccia testuale bisogna
aggiungere l'opzione \texttt{-nw} oppure \texttt{--no-window-system}:
\begin{Verbatim}
$ emacs -nw
\end{Verbatim}
% $
Possono essere aggiunti come argomenti da linea di comando il percorso
del file (o dei file) che si vogliono modificare:
\begin{Verbatim}
$ emacs file1.tex file2.tex
\end{Verbatim}
% $


\section{Installare AUC\TeX}
\label{mg:sec:installauctex}

Anche il pacchetto \auctex{} è presente nei sistemi GNU/Linux. Come già detto
per \emacs, per installare \auctex{} si consiglia di utilizzare il gestore
pacchetti della propria distribuzione. Ecco i comandi da usare per installare da
terminale il pacchetto nelle principali distribuzioni: Debian e Ubuntu:
\begin{Verbatim}
$ sudo apt-get install auctex
\end{Verbatim}
% $
Fedora:
\begin{Verbatim}
$ sudo yum install emacs-auctex
\end{Verbatim}
% $
OpenSUSE:
\begin{Verbatim}
$ sudo zypper install emacs-auctex
\end{Verbatim}
% $
Su Debian e Ubuntu potrebbe essere raccomandata l'installazione della
distribuzione TeX Live presente nei repository ufficiali del sistema operativo.
Per evitare che questo avvenga, per esempio perché si utilizza una distribuzione
TeX Live installata in altro modo, se si installa il pacchetto da terminale è
sufficiente aggiungere l'opzione \verb!--no-install-recommends!
\begin{Verbatim}
$ sudo apt-get install --no-install-recommends auctex
\end{Verbatim}
% $


\section{Correttore ortografico}
\label{mg:sec:aspell}

Il correttore ortografico GNU Aspell può essere installato facilmente
su GNU/Linux utilizzando, come al solito, il gestore pacchetti della
propria distribuzione. Il dizionario italiano di GNU Aspell si chiama
\texttt{aspell-it} quindi da terminale può essere installato in Debian
con il comando
\begin{Verbatim}
$ sudo apt-get install aspell-it
\end{Verbatim}
% $
in Fedora con
\begin{Verbatim}
$ sudo yum install aspell-it
\end{Verbatim}
% $
mentre in openSUSE si può eseguire il comando:
\begin{Verbatim}
$ sudo zypper install aspell-it
\end{Verbatim}
% $


\section{Ricerca diretta inversa}
\label{mg:sec:ricdirinv}

\textcolor{red}{\lipsum[2]}

%% --------------------------------------------------------------------------
%% Mac OS
%% --------------------------------------------------------------------------

\chapter{Emacs e AUC\TeX{} su Mac OS}
\label{ml:chap:linux}

\epigraph{(\textit{a cura di} Matteo Leccardi)}

\section{Installare Emacs}
\label{ml:sec:installemacs}

Sul mirror ufficiale %
\sito{http://ftp.gnu.org/pub/gnu/emacs/windows/}{\emacs} non
è presente una versione già compilata per Mac OS X, queste versioni
sono rese disponibili da vari volontari come ad esempio il gestore del
sito \sito{http://emacsformacosx.com/}{Emacs for Mac OS X}. La
procedura per l'installazione è quella consueta: scaricata la versione
più recente si apre il file dmg (se non è già stato aperto
automaticamente al termine del download) e si trascina Emacs nella
cartella Applicazioni.

In Mac OS le applicazioni avviate dall'interfaccia grafica non hanno
accesso ai valori delle variabili d'ambiente, per rendere disponibile
ad \emacs{} la variabile \texttt{PATH} è necessario dare da terminale
il comando
\begin{Verbatim}
$ defaults write ~/.MacOSX/environment PATH "$PATH"
\end{Verbatim}
Le nuove impostazioni sono effettive dopo aver eseguito un logout e un
successivo login.

Il comando va ripetuto ogni volta che si installa qualche programma
che modifica il valore di \texttt{PATH}, il caso che più probabilmente
riguarda i lettori di questa guida è la distribuzione \TeX{}.

% TODO: controllare che i tasti inseriti con `\keys{}' siano corretti
Per gli utenti che usano la tastiera italiana è utile impostare il
tasto \keys{\cmdmac} come \keys{\Meta} e il tasto \keys{\Altmac} per
scrivere i caratteri speciali, in particolare le parentesi quadre, le
graffe e la tilde, scrivendo queste righe nel file \emph{.emacs}
\begin{Verbatim}
(setq ns-command-modifier 'meta)
(setq ns-alternate-modifier nil)
\end{Verbatim}


\section{Installare AUC\TeX}
\label{ml:sec:installauctex}

Come per \emacs{} non è disponibile una versione di \auctex{} già compilata per
Mac OS, in questo caso non esistono neppure versioni non ufficiali ed è quindi
necessario utilizzare l'utility \texttt{make}.  \texttt{make} viene installato
insieme all'ambiente di sviluppo
\href{http://developer.apple.com/technologies/tools/whats-new.html}%
{Xcode}, chi non intende dedicare svariati GB di disco a un software utile
soltanto a chi sviluppa software può sfuttare il fatto che \texttt{make} viene
installato anche dal correttore ortografico
\href{http://cocoaspell.leuski.net/}{cocoaspell} la cui installazione è
descritta nel paragrafo \ref{ml:sec:aspell}.

Dalla pagina relativa a Mac OS
\sito{http://www.gnu.org/software/auctex/download-for-macosx.html}%
{\auctex\ For Mac OS X} bisogna scaricare l'archivio
\href{http://ftp.gnu.org/pub/gnu/auctex/auctex-11.86.tar.gz}%
{auctex-11.86.tar.gz}.

Dopo aver estratto il contenuto dell'archivio in una cartella temporanea occorre
aprire una sessione di terminale e portarsi nella cartella appena creata.

Configurare \auctex{} con il comando
\begin{Verbatim}
$ ./configure --prefix=/Applications/Emacs.app/Contents/Resources/\
  --with-emacs=/Applications/Emacs.app/Contents/MacOS/Emacs\
  --with-lispdir=/Applications/Emacs.app/Contents/Resources/site-lisp/\
  --without-texmf-dir
\end{Verbatim}
% $
sostituendo se necessario a \texttt{Applications}\footnote{Utilizzando il
  terminale non bisogna utilizzare i nomi localizzati delle cartelle ma quelli
  reali, quindi \texttt{Application} è corretto anche se si utilizza la versione
  italiana di Mac OS} la cartella in cui si trova \emacs{}, in questo modo tutti
i file saranno contenuti nel bundle \texttt{Emacs.app}.

Compilare e installare con il comando
\begin{Verbatim}
$ make && make install
\end{Verbatim}
%$

Dopo l'installazione bisogna impostare i visualizzatori per i vari file creati
con \LaTeX{} scrivendo nel file \emph{.emacs} queste righe
\begin{Verbatim}
(setq TeX-view-program-list
      '(("dvips and Skim" "%(o?)dvips %d -o &&\
  /Applications/Skim.app/Contents/SharedSupport/displayline %n %f %b")
         ("Skim" "/Applications/Skim.app/Contents/SharedSupport/\
displayline %n %o %b")
        ("open" "open %o")))
(setq TeX-view-program-selection
      '(((output-dvi style-pstricks) "dvips and Skim")
        (output-dvi "Skim")
        (output-pdf "Skim")
        (output-html "open")))
\end{Verbatim}
Come visualizzatore principale si utilizza
\href{http://skim-app.sourceforge.net/}{Skim} perché supporta la ricerca diretta
e inversa dal pdf, i dettagli sulla configurazione di \emacs{} e Skim sono
riportati nel paragrafo~\ref{ml:sec:ricdirinv}.


\section{Correttore ortografico}
\label{ml:sec:aspell}

Anche per Mac OS X è disponibile una versione di \textsf{GNU Aspell}
\sito{http://cocoaspell.leuski.net/}{cocoAspell}

Dopo aver installato \href{http://people.ict.usc.edu/~leuski/%
  cocoaspell/cocoAspell.2.1.dmg}{cocoAspell.2.1} si possono installare
i dizionari delle lingue che interessano scaricandoli direttamente dal
sito di (\href{ftp://ftp.gnu.org/gnu/aspell/dict/}{\textsf{GNU
    Aspell}}), ad esempio il dizionario italiano è contenuto
nell'archivio
\href{ftp://ftp.gnu.org/gnu/aspell/dict/it/aspell6-it-2.2_20050523-0.tar.bz2}%
{aspell6-it-2.2\_20050523-0.tar.bz2}.

Dopo aver scaricato i dizionari bisogna estarre il contento
dell'archivio in una cartella temporanea, aprire una sessione di
terminale e portarsi nella cartella appena creata. I dizionari si
installano dando i seguenti comandi
\begin{Verbatim}
$ ./configure
$ make
$ sudo make install
\end{Verbatim}
% $

Come ultima cosa è necessario configurare \emacs\ aggiungendo al
\emph{.emacs} le seguenti righe:
\begin{Verbatim}
(setq-default ispell-program-name "aspell")
(setq ispell-dictionary "italiano")
\end{Verbatim}


\section{Ricerca diretta inversa}
\label{ml:sec:ricdirinv}

L'unico visualizzatore di file PDF per Mac OS X che supporta la
ricerca diretta e inversa associato a \emacs{} è
\href{http://skim-app.sourceforge.net/}{Skim}, già citato nel
paragrafo~\ref{ml:sec:installauctex}, relativo all'installazione di
\auctex{}.

Dopo averlo installato occorre configurarlo impostando quanto segue
nella pagina \emph{Sincronizza} delle preferenze:
\begin{itemize}
\item selezionare la checkbox \emph{Controlla cambiamenti del file}
\item scrivere
  \verb!/Applications/Emacs.app/Contents/MacOS/bin/emacsclient!  nel
  campo \emph{Comando}
\item scrivere \verb!--no-wait +%line "%file"! nel campo
  \emph{Argomenti}
\end{itemize}
Nella figura~\ref{fig:skimpref} sono riportate le impostazioni
corrette.
\begin{figure}[tb]
  \centering
  \includegraphics[width=\textwidth]{preferenze_skim}
  \caption{Preferenze di Skim per la ricerca diretta e inversa}
  \label{fig:skimpref}
\end{figure}

La prima opzione fa sì che quando un file aperto in Skim viene
modificato da un altro processo Skim chieda se deve essere ricaricato,
scegliendo Automatico successivi aggiornamenti vengono ricaricati
senza chiedere conferma.

La configurazione di \emacs{} è analoga a quella degli altri sistemi
operativi, occorre scrivere queste righe in \emph{.emacs}
\begin{Verbatim}
(setq TeX-source-correlate-method 'synctex)
(setq TeX-source-correlate-mode t)
(setq TeX-source-correlate-start-server t)
\end{Verbatim}

Per passare dal file .tex al .pdf si utilizza la combinazione di tasti
\verb!C-c C-v!, per ritornare al file .tex bisogna fare click sulla
parola che interessa tendendo premuto shift e cmd.

Con alcune versioni di \emacs{} precedenti alla 23.3 resta in primo
piano la finestra del visulizzatore, in tal caso è necessario
aggiungere queste righe al file \emph{.emacs}
\begin{Verbatim}
(defun ns-raise-emacs ()
  (ns-do-applescript "tell application \"Emacs\" to activate"))
(add-hook 'server-switch-hook 'ns-raise-emacs)
\end{Verbatim}



\chapter{Personalizzare \emacs{} \& \auctex}
\label{chap:personal}

In questo paragrafo vedremo alcune delle personalizzazioni che possono risultare
utili quando si lavora con documenti \LaTeX{}.  Per modificare le impostazioni
relative alla gestione nativa di \TeX{} da parte di \emacs{} è possibile usare
la funzione \verb!M-x customize-group RET tex RET!.  In questo modo si aprirà un
buffer nel quale sarà possibile modificare, tramite l'interfaccia, le opzioni
desiderate.  Per modificare in particolare \auctex{} è possibile usare la
funzione \verb!M-x customize-group RET AUCTeX RET!.

Di seguito saranno suggerite alcune porzioni di codice Elisp che permettono di
personalizzare \emacs{} e che vanno inserite nel file di inizializzazione
\verb!~/.emacs!.  Per rendere effettive le modifiche bisogna riavviare \emacs{}.

Se si usa \auctex{}, è possibile inserire macro nei propri documenti \LaTeX{}
con \verb!C-c RET!.  La comodità di questa funzione è che \auctex{} conosce le
macro dei principali pacchetti \LaTeX{} e permette quindi l'autocompletamento
con il tasto TAB.  Inoltre conosce gli argomenti e le opzioni che queste macro
accettano, infatti usando \verb!C-c RET frac RET! nel documento si otterrà
\verb!\frac{}{}!  e il cursore si posizionerà all'interno del primo paio di
parentesi graffe.  Allo stesso modo, usando \verb!C-c RET sqrt RET! verrà
richiesto l'ordine \verb!n! della radice da inserire (premere direttamente
\verb!RET! per non inserire nulla) e si otterrà \verb!\sqrt[n]{}!.  Per fare in
modo che \auctex{} conosca quali pacchetti \LaTeX{} sono stati caricati bisogna
permettergli di effettuare il parsing dei propri documenti e questo è possibile
aggiungendo nel proprio file \verb!.emacs! il seguente codice
\begin{Verbatim}
(setq TeX-parse-self t) ; Attiva parsing al caricamento.
(setq TeX-auto-save  t) ; Attiva parsing al salvataggio.
\end{Verbatim}
In questo modo verrà creata una cartella chiamata \verb!auto! nella stessa
directory in cui si trova il documento nella quale verranno registrate le
informazioni relative al proprio documento \LaTeX{}, se non si desidera
affollamento nelle proprie cartelle si potrebbe non voler attivare queste due
opzioni.

Come detto, \auctex{} conosce le macro dei pacchetti principali, ma naturalmente
non conosce tutte i pacchetti.  Si può far in modo che \auctex{} analizzi tutti
i file di stile della propria distribuzione \LaTeX{} eseguendo il comando
\verb!M-x TeX-auto-generate-global!.  In questo modo verrà automaticamente
aggiunto il supporto a tutti i pacchetti, almeno quelli non troppo complicati e
che non siano basati su \LaTeX{}3.  L'operazione può richiedere alcuni minuti,
soprattutto la prima volta che viene eseguito.  Per i pacchetti non
correttamente analizzati da \auctex{}, possiamo aggiungere manualmente le macro
da utilizzare solo quando viene caricato nel documento \LaTeX{} il pacchetto che
fornisce le relative macro.  Per esempio, uno dei pacchetti basati su \LaTeX{}3
è \verb!siunitx!.  Con il seguente codice da aggiungere nel file \verb!.emacs!
apprenderà che questo pacchetto fornisce i comandi \verb!SI!, \verb!si!,
\verb!ang! e \verb!num! di cui il primo accetta due argomenti, gli ultimi tre
solo uno:
\begin{Verbatim}
(eval-after-load "tex"
  '(TeX-add-style-hook
    "siunitx"
    (lambda ()
      (TeX-add-symbols
       '("SI" 2)
       '("si" 1)
       '("ang" 1)
       '("num" 1)))))
\end{Verbatim}
A volte in un documento \LaTeX{} non si carica un certo pacchetto \verb!B!
poiché è automaticamente caricato da un altro pacchetto \verb!A! che viene
esplicitamente caricato.  In questi casi, però, \auctex{} potrebbe non rendersi
conto che il pacchetto \verb!B! è stato effettivamente caricato e non fornirebbe
le funzioni di autocompletamento per le macro del pacchetto \verb!B!, se è uno
di quelli supportati.  Supponiamo inoltre che insieme a \verb!B!, il pacchetto
\verb!A! carichi anche i pacchetti \verb!C! e \verb!D!.  Per ovviare a questo
problema è possibile inserire il seguente codice nel file di inizializzazione
\begin{Verbatim}
(eval-after-load "tex"
  '(TeX-add-style-hook
    "A"
    (lambda ()
      (TeX-run-style-hooks "B" "C" "D"))))
\end{Verbatim}
Per esempio, il pacchetto \verb!siunitx! carica automaticamente i pacchetti
\verb!array! e \verb!amstext! e non è quindi necessario inserire nel proprio
sorgente \LaTeX{} le istruzioni \verb!\usepackage{array}! e
\verb!\usepackage{amstext}!.  Per far sapere ad \auctex{} questa situazione
possiamo completare il codice precedente come segue
\begin{Verbatim}
(eval-after-load "tex"
  '(TeX-add-style-hook
    "siunitx"
    (lambda ()
      (TeX-add-symbols
       '("SI" 2)
       '("si" 1)
       '("ang" 1)
       '("num" 1)
      (TeX-run-style-hooks "array" "amstext")))))
\end{Verbatim}

Se si è soliti suddividere i propri documenti \LaTeX{} in più file da includere
nel file principale con i comandi \verb!\input! o \verb!\include! può essere
utile aggiungere al file \verb!.emacs!
\begin{Verbatim}
(setq-default TeX-master nil)
\end{Verbatim}
In questo modo, all'apertura dei documenti \LaTeX{} \emacs{} chiederà qual è il
file principale associato e sul quale saranno eseguiti i comandi di
compilazione.

Se si usa spesso \LaTeX{} per comporre documenti matematici potrebbe essere
utile attivare la modalità \verb!LaTeX-math-mode! con \verb!C-c ~! oppure con il
comando \verb!M-x LaTeX-math-mode!.  In questo modo, per esempio, per inserire
il simbolo \verb!\alpha! si potrà usare la scorciatoia \verb!` a!.  L'elenco
delle altre scorciatoie per inserire i simboli può essere consultato dal menu
\verb!Math! nella barra dei menu.  Per attivare automaticamente questa modalità
ogni volta che si apre un documento \LaTeX{} bisogna aggiungere
\begin{Verbatim}
(add-hook 'TeX-mode-hook 'LaTeX-math-mode)
\end{Verbatim}
al proprio \verb!.emacs!.

Il codice
\begin{Verbatim}
(setq TeX-electric-sub-and-superscript t)
\end{Verbatim}
aggiunto al proprio \verb!.emacs!, permette di inserire automaticamente una
coppia di parentesi graffe \verb!{}! quando si scrivono i simboli \verb!_! e
\verb!^! in modalità matematica.

Normalmente, quando si preme \verb!RET! in un sorgente \LaTeX{} viene
semplicemente aggiunta una nuova linea.  Se si vuole che andando a capo la nuova
linea sia automaticamente indentata bisogna premere \verb!C-j! invece di
\verb!RET!.  Si può però fare in modo che si ottenga questo risultato anche
premendo normalmente \verb!RET! aggiungendo il seguente codice al file di
inizializzazione
\begin{Verbatim}
(setq TeX-newline-function 'newline-and-indent)
\end{Verbatim}
Se si mette \verb!reindent-then-newline-and-indent! al posto di
\verb!newline-and-indent! si avrà, in più, l'effetto che la riga attuale viene
indentata.

\chapter{Conclusioni}
\label{chap:fine}

\textcolor{red}{\lipsum[1]}

\nocite{*}
\bibliography{introemacs}

\end{document}

%%% Local Variables:
%%% mode: latex
%%% coding: utf-8
%%% TeX-master: t
%%% fill-column: 80
%%% End:

% LocalWords:  mirror OS For dmg Git Fast Version Control System clonare git
% LocalWords:  clone



%%%%%%%%%%%%%%%%%%%%%%%%%%%%%%%%%%%%%%%%%%%%%%%%%%%%%%%%%%%%%%%%%%%%%%%%%%%%%%%
%%%%%%%%%%%%%%%%%%%%%%%%%%%%%%%%%%%%%%%%%%%%%%%%%%%%%%%%%%%%%%%%%%%%%%%%%%%%%%%
%%%%%%%%%%%%%%%%%%%%%%%%%%%%%%%%%%%%%%%%%%%%%%%%%%%%%%%%%%%%%%%%%%%%%%%%%%%%%%%

%% TROVARE UN POSTO DOVE SISTEMARE QUESTE COSE (SE UTILI).

%%%%%%%%%%%%%%%%%%%%%%%%%%%%%%%%%%%%%%%%%%%%%%%%%%%%%%%%%%%%%%%%%%%%%%%%%%%%%%%

% \chapter{Installare \auctex}
% \label{chap:installauc}
% A partire dalla versione 24.1 di \emacs{} si possono installare i pacchetti di
% estensione all'interno di \emacs{} stesso. Basta infatti eseguire
% \verb!M-x list-packages! e verranno visualizzati tutti i pacchetti disponibili,
% installati e non. \auctex, quindi, può essere installato in questo modo,
% premendo semplicemente su \emph{Install} dopo averlo selezionato.
% % TODO (Mosè 30/10/2012): immagino che se si segue questa procedura sia
% % necessario aggiungere una qualche cartella al `load-path' con
% %       (add-to-list 'load-path "<PERCORSO>")
% % e poi caricare `auctex' e `preview-latex' con
% %       (load "auctex.el" nil t t)
% %       (load "preview-latex.el" nil t t)
% % Non ho ancora Emacs 24, ancora non so quale sia la procedura completa.

% Per installare \auctex{} in una versione di \emacs{} precedente alla 24.1 è
% necessario seguire una procedura diversa a seconda del sistema operativo
% utilizzato.  Di seguito sono riportate le istruzioni da seguire in questo caso
% per Windows, GNU/Linux e Mac OS.

%%%%%%%%%%%%%%%%%%%%%%%%%%%%%%%%%%%%%%%%%%%%%%%%%%%%%%%%%%%%%%%%%%%%%%%%%%%%%%%

% \chapter{Correttore ortografico}
% \label{chap:corr}

% Per abilitare \emacs{} e \auctex{} a lavorare con la lingua italiana bisogna
% installare il correttore ortografico. Le procedure, poiché differenti, vengono
% di seguito esposte per tutti e tre i sistemi operativi.  L'uso, invece, è
% invariante.

% Per fare una revisione del documento che si sta scrivendo, bisogna premere la
% combinazione di tasti \verb!M-x! e scrivere \verb!ispell!: in questo modo il
% dizionario proporrà, in sequenza, la correzione dei termini che trova errati.

% Un secondo modo, per scrivere correttamente, è attivare la modalità
% \emph{flyspell} (correzione al volo). Se la si vuole attivare solo sul documento
% corrente basta andare in \menu{Tools > Spell Checking > Automatic spell
%   checking (Flyspell)}; se invece la si vuole tenere sempre abilitata si deve
% aggiungere la riga:
% \begin{Verbatim}
% (add-hook 'LaTeX-mode-hook 'flyspell-mode)
% \end{Verbatim}
% al proprio file \emph{.emacs}.

% Quando viene rilevata una parola potenzialmente non corretta, questa verrà
% colorata (in genere) in rosso. Alcuni termini, però, vengono segnalati come
% errati anche quando sono corretti, cosa che accade perché quel termine non è
% presente nel dizionario: si può aggiungerlo, premendo il tasto centrale del
% mouse quando questo è posizionato sulla parola e scegliere \texttt{Save word}.

%%%%%%%%%%%%%%%%%%%%%%%%%%%%%%%%%%%%%%%%%%%%%%%%%%%%%%%%%%%%%%%%%%%%%%%%%%%%%%%